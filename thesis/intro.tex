
\chapter{Introduction}
	Start with a motivation.
	What is the problem?
	What is it part of?
	What is the exact problem you want to solve in your work?
	What is the benefit of solving it?
	How do you solve it (sketch only)?

% 	\section{Research Question}
% 		%  1. a concise statement of the question that your thesis tackles 
% 		%  2. justification, by direct reference to section 3 that your question is previously unanswered
% 		%  3. discussion of why it is worthwhile to answer this question.
	
	\section{About This Skeleton}
		This skeleton tries to guide you in finding a good structure for your thesis and help you to represent it in \LaTeX.
		
		You should be aware of the fact that each field of computer science comes with its own ``default structure'' for a thesis.
		The default structure of your thesis has a very central purpose for your reader: quickly finding the relevant parts and getting an overview.
		I heard arguments for not following this default structure, because it is boring.
		I don't care, it's not a novel or a story you're writing, you're focus lies on quickly exposing information to your reader.
		
		If I was to read (not necessarily grade) your thesis I would read it like this (pretty much like a paper):
		\begin{description}
			\item[Abstract] (What is it about? What are the central ideas? Do I care/need this?)
			\item[Table of contents] (Short glimpse where to expect what)
			\item[Intro] (Search for main contribution and ``why?'')
			\item skip over the thesis in large steps while going to the conclusion (Just for orientation)
			\item[Conclusion] (Does it work, problems solved, future work, discussion)
			\item[References] (Do you cite key papers in the area?)
			\item[Evaluation] (Is the evaluation sound? does it really work?)
			\item[Related work] (What is related? how does your approach differ?)
			\item[Main parts] (Details on how you solved things)
			\item[Foundations] (Only if referred to from main parts and things are unknown to me)
		\end{description}
		In any of these steps i would stop if i lost my interest.
		This probably doesn't count for someone grading your thesis as (s)he has to go on anyhow, but probably these still are the first steps (and the last two are reading the work completely).
		
		As mentioned before every subfield of computer science seems to have its own structure.
		Math works look quite different from software engineering works and implementation works look different as well.
		Still the main structure can often be found maybe with different names (e.g., evaluation might be called complexity analysis...).
		
		So let yourself be guided by this skeleton, try to find something for each of the chapters and possibly rename them according to your needs.
		And last but not least: 
		Make sure to check back with your supervisors in an early stage e.g., after a first ``extended outline'' (mainly chapters, sections, subsections and \verb|\redline|s), before wasting a lot of time for nothing.
	
	\section{Structure of this Thesis}\label{sec:structure} 
% 		\redline{\item Figure~\ref{fig:ProcessOverview}
% 					\item Input: Data Acquisition and Preprocessing
% 					\item The Game: Getting player preferences
% 					\item Output: Generate a Ranking from acquired preferences}
% 		\begin{figure}
% 			\centering
% 			\includegraphics[width=\textwidth]{./pics/processOverview}
% 			\caption{Data Flow Process Overview}
% 			\label{fig:ProcessOverview}
% 		\end{figure}
		
		Briefly explain the structure of the thesis.
		If you have a process overview, include a picture showing the whole thing (better than a thousand words).
		Explain which part of the picture is explained where leading do your evaluation and conclusion.
		
